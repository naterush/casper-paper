% make in the latest NIPS format (as of this writing, 2017)

\usepackage[nonatbib,final]{nips_2017}
%\usepackage[nonatbib,final]{nips_2017}

\usepackage{color}
\usepackage{graphicx}
\DeclareGraphicsExtensions{.pdf,.eps,.png,.jpg}     % search for .pdf, then .eps, then .pngs, then .jpg

% look in these subdirectories for graphics referenced by \includegraphics
% each entry must end with a /
\graphicspath{{figs/}{figures/}{images/}{./}}

\newcommand*{\red}[1]{ \color{red} #1}

%\usepackage{tabularx}
\usepackage{url}
\usepackage{amsmath}

% this is for environments \subfigure and \subtable
\usepackage{subcaption}

% These packages are FORBIDDEN
%%%%%%%%%%%%%%%%%%%%%%%%%%%%%%%%%%%%%%%%%%%%%%%%%%%%%%%%
%\usepackage{lmodern} % messes up \textsc
%\usepackage{cite} % messes up NIPS
%\usepackage{fullpage} % messes up NIPS
%\usepackage{natbib} % messes up NIPS
%%%%%%%%%%%%%%%%%%%%%%%%%%%%%%%%%%%%%%%%%%%%%%%%%%%%%%%%

\usepackage{array}          % replacement for eqnarray.  Must be BEFORE \usepackage{arydshln}
\usepackage{units}          % for \nicefrac{\alpha}{\beta}


\usepackage{amsthm}     % for theorems
\newtheorem{definition}{Definition}

% text looks a little better
\usepackage{microtype}

\usepackage{wasysym}

\usepackage{textcomp, marvosym} % pretty symbols
\usepackage{booktabs}   % for much better looking tables

% for indicator functions
\usepackage{dsfont}

% For automatic capitalizaton of section names, etc.
\usepackage{titlesec,titlecaps}


\Addlcwords{is with of the and in}
\Addlcwords{of the}
\Addlcwords{and}
\titleformat{\section}[block]{}{\normalfont\Large\bfseries\thesection.\;}{0pt}{\formatsectiontitle}
\newcommand{\formatsectiontitle}[1]{\normalfont\Large\bfseries\titlecap{#1}}

\titleformat{\subsection}[block]{}{\normalfont\large\bfseries\thesubsection.\;}{0pt}{\formatsubsectiontitle}
\newcommand{\formatsubsectiontitle}[1]{\normalfont\large\bfseries\titlecap{#1}}





% for pretty Euler script
% \usepackage[mathscr]{euscript}
% \usepackage{bold-extra}





%\usepackage{subfig}
\usepackage{float} % for \subfloat

%%%%%%%%%%%%%%%%%%%%%%%%%%%%%%%%%%%%%%%%%%
%% More customizable Lists
%%%%%%%%%%%%%%%%%%%%%%%%%%%%%%%%%%%%%%%%%%
% Better symbols custom enumerative lists, define any symbol you'd like
% \usepackage{enumitem}


%%%%%%%%%%%%%%%%%%%%%%%%%%%%%%%%%%%%%%%%%%
%% Custom Symbols 
%%%%%%%%%%%%%%%%%%%%%%%%%%%%%%%%%%%%%%%%%%
% \xspace at the end of custom macros never screws up spacing.
\usepackage{xspace}



%%%%%%%%%%%%%%%%%%%%%%%%%%%%%%%%%%%%%%%%
%% Abbreviations you'll always want
%%%%%%%%%%%%%%%%%%%%%%%%%%%%%%%%%%%%%%%%
\newcommand*{\TODO}[1]{{\centering {\small \sffamily \color{red} #1} \vskip10pt }}
\newcommand*{\todo}[1]{{\small \sffamily [{\color{red} #1}]}}
\newcommand*{\q}[1]{{\small \sffamily [{\color{blue} #1}]}}
\newcommand*{\fix}[1]{{\sffamily [{\textnormal{\color{red} #1}}]}}



%-----------------------------------------------------------------------------
%  Cross references
%-----------------------------------------------------------------------------
% The following code defines how you make references to figures, tables, etc...
% It is defined in one place only, and can be modified for all references
% in the document at the same time.
% Instead of typing each time: "see Fig. \ref{myfig}" you can create a command
% \figref which will do the job. Then in text you only type \figref{myfig} and LaTeX
% will do the rest.
\newcommand{\tblref}[1]{Table~\ref{#1}}
%\renewcommand*{\figref}[1]{Fig.~\ref{#1}}
\renewcommand{\eqref}[1]{eq.~(\ref{#1})}
\newcommand{\Subref}[1]{(\subref{#1})}


\newcommand{\figref}[1]{Figure~\ref{#1}}
\newcommand{\Figref}[1]{Figure~\ref{#1}}
\newtheorem{theorem}{Theorem}
\newtheorem{lemma}[theorem]{Lemma}

%%%%%%%%%%%%%%%%%%%%%%%%%%%%%%%%%%%%%%%%

% for \sout{} for strikeout
% \usepackage[normalem]{ulem}


% for better manipulation of tables
\usepackage{makecell}
\renewcommand\theadfont{\bfseries}


%-----------------------------------------------------------------------------
%  Misc symbols that I like
%-----------------------------------------------------------------------------
\newcommand*{\opname}[1]{\operatorname{#1}}


\renewcommand*{\to}{\rightarrow}


%%%%


